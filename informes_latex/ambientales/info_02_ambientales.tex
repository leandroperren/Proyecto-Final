\documentclass{article}
% pre\'ambulo

\usepackage{lmodern}
\usepackage[T1]{fontenc}
\usepackage[spanish,activeacute]{babel}

\title{Informe de resultados: se�ales reales}
\author{Leandro Perren - Matias Meurzet}

\begin{document}
% cuerpo del documento

\maketitle

\section{Introducci�n} 
En este documento se presenta los resultados de obtener se�ales grabadas en ambientes reales cotidianos, tanto de ronquidos como de otros sonidos (aqu� los definiremos como ruidos) que no son ronquidos y que pueden estar presentes en las se�ales obtenidas en entornos reales cotidianos, como por ejemplo:

\begin{enumerate}
\item Ambientales
\item Tos, estornudos
\item Veh�culos
\item Voz
\item M�sica
\end{enumerate}

El objetivo es poder evaluar caracteristicas temporales (energia, zcr, rms, factor cresta) en las se�ales grabadas en entornos reales cotidianos.  
Una vez extra�das las caracter�sticas necesarias se podr�, entrenar y evaluar distintos modelos/clasificadores con la intenci�n de construir una aplicaci�n para dispositivos m�viles capaz de detectar episodios de ronquidos y diferenciarlo de otras interferencias (tos, veh�culos, voces, m�sica, etc.) que pudieran acoplarse en la grabaci�n del sue�o de la persona.

\section{Procedimiento}
\end{document}